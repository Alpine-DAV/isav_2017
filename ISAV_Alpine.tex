\documentclass[sigconf]{acmart}

\usepackage{booktabs} % For formal tables
\usepackage{listings}

% Copyright
%\setcopyright{none}
%\setcopyright{acmcopyright}
%\setcopyright{acmlicensed}
\setcopyright{rightsretained}
%\setcopyright{usgov}
%\setcopyright{usgovmixed}
%\setcopyright{cagov}
%\setcopyright{cagovmixed}


% DOI
\acmDOI{10.475/123_4}

% ISBN
\acmISBN{123-4567-24-567/08/06}

%Conference
\acmConference[ISAV 2017]{ISAV 2017: In Situ Infrastructures for Enabling Extreme-scale Analysis and Visualization}{November 2017}{Denver, Colorado USA} 
\acmYear{2017}
\copyrightyear{2017}

\acmPrice{15.00}


\begin{document}
\title{The ALPINE In Situ Infrastructure: \\ Rising from the Ashes of Strawman}
\titlenote{Produces the permission block, and
  copyright information}
\subtitle{Middle authors in alphabetical order:}
\subtitlenote{The full version of the author's guide is available as
  \texttt{acmart.pdf} document}

\author{Matthew Larsen}
\affiliation{\institution{Lawrence Livermore National Lab}}
\email{larsen30@llnl.gov}

\author{James Ahrens}
\affiliation{\institution{Los Alamos National Lab}}
\email{ahrens@lanl.gov}

\author{Utkarsh Ayachit}
\affiliation{\institution{Kitware, Inc}}
\email{utkarsh.ayachit@kitware.com}

\author{Eric Brugger}
\affiliation{\institution{Lawrence Livermore National Lab}}
\email{brugger1@llnl.gov}

\author{Hank Childs}
\affiliation{\institution{University of Oregon}}
\email{hank@uoregon.edu}

\author{Berk Geveci}
\affiliation{\institution{Kitware, Inc}}
\email{berk.geveci@kitware.com}

\author{Cyrus Harrison}
\affiliation{\institution{Lawrence Livermore National Lab}}
\email{cyrush@llnl.gov}



% The default list of authors is too long for headers}
\renewcommand{\shortauthors}{Humans et al.}



\begin{abstract}
This paper introduces ALPINE, a flyweight in situ infrastructure.
%
The infrastructure is designed for leading-edge supercomputers, and
has support for both distributed-memory and shared-memory parallelism.
%
It can take advantage of computing power on both conventional CPU architecture
and on many-core architectures such as NVIDIA GPUs or the Intel Xeon Phi.
%
Further, it has a flexible design that allows support for integrating new
visualization and analysis routines and libraries.
%
The paper describes ALPINE's interface choices and architecture, and also reports
on initial experiments performed using the infrastructure.
\end{abstract}

%
% The code below should be generated by the tool at
% http://dl.acm.org/ccs.cfm
% Please copy and paste the code instead of the example below. 
%
\begin{CCSXML}
<ccs2012>
 <concept>
  <concept_id>10010520.10010553.10010562</concept_id>
  <concept_desc>Computer systems organization~Embedded systems</concept_desc>
  <concept_significance>500</concept_significance>
 </concept>
 <concept>
  <concept_id>10010520.10010575.10010755</concept_id>
  <concept_desc>Computer systems organization~Redundancy</concept_desc>
  <concept_significance>300</concept_significance>
 </concept>
 <concept>
  <concept_id>10010520.10010553.10010554</concept_id>
  <concept_desc>Computer systems organization~Robotics</concept_desc>
  <concept_significance>100</concept_significance>
 </concept>
 <concept>
  <concept_id>10003033.10003083.10003095</concept_id>
  <concept_desc>Networks~Network reliability</concept_desc>
  <concept_significance>100</concept_significance>
 </concept>
</ccs2012>  
\end{CCSXML}

\ccsdesc[500]{Computer systems organization~Embedded systems}
\ccsdesc[300]{Computer systems organization~Redundancy}
\ccsdesc{Computer systems organization~Robotics}
\ccsdesc[100]{Networks~Network reliability}


\keywords{ACM proceedings, \LaTeX, text tagging}

\newcommand{\fix}[1]{\textcolor{red}{#1}}

\lstset{language=C++,
	basicstyle=\ttfamily,
	showstringspaces=false,
	keywordstyle=\color{blue}\ttfamily,
	stringstyle=\color{red}\ttfamily,
	commentstyle=\color{gray}\ttfamily,
	morecomment=[l][\color{magenta}]{\#}
}


\maketitle

\input{intro.inc}
\input{related_work.inc}
\input{interface.inc}

\input{architecture.inc}
\input{results.inc}

\bibliographystyle{ACM-Reference-Format}
\bibliography{ISAV_Alpine} 

\end{document}
